\documentclass[11pt]{article}

\usepackage{minted,xcolor}
\usemintedstyle{monokai}
\definecolor{bg}{HTML}{282828}
\usepackage{xcolor, colortbl}
\usepackage{graphicx}
\usepackage{titlesec}
\usepackage[hmargin=2cm,vmargin=2.5cm]{geometry}
\usepackage{fancyhdr}
\usepackage{hyperref}
\usepackage{caption}
\usepackage{subcaption}

\definecolor{tableaublue}{RGB}{78, 121, 167}

\hypersetup{
    colorlinks=true,
    linkcolor=black,
    filecolor=tableaublue,      
    urlcolor=tableaublue,
    pdftitle={Overleaf Example},
    pdfpagemode=FullScreen,
}

\setlength\parindent{0cm}
\setlength\headheight{28pt}

\titleformat{\section}{\normalfont\Large\bfseries}{Part~\thesection:}{6pt}{}[{\titlerule[0.5pt]}]
\titleformat{\subsection}{\normalfont\large\bfseries}{Exercise~\thesubsection:}{6pt}{}

\graphicspath{ {./img/} }

% \title{Practical’s instruction – Networks (week 7)}
% \author{CSC3833 - Data Visualization and Visual Analytic}

\begin{document}

\pagestyle{fancy}
\renewcommand{\headrulewidth}{0pt}
\fancyhead[L]{CSC8626/CSC8642 - Data Visualization}
\fancyhead[R]{2023 / 2024}
\fancyfoot[L]{\thepage}
\fancyfoot[C]{}
\fancyfoot[R]{School of Computing, Newcastle University}

\begin{center}
\vspace*{1cm}
{\textbf {\Huge Practical 4}}\\
\vspace*{0.5cm}
{\textbf {\huge Human Perception and Graphic Design}}
\vspace*{1cm}
\end{center}

\section{Part 1}

Part one of this assignment is about building a narrative to convey information about displays resolution and the human visual system.

\subsection*{Data}

\underline{Files}: CalculatedResolutions.xlsx\\
These files are available on the Canvas page of the module, which can be accessed at \href{https://ncl.instructure.com/courses/49730}{link}.\\

\underline{Parameters}:
\begin{table}[h!]
    \centering
    \begin{tabular}{|l|m{10cm}|}
        \hline
        Display & type of display (tablet, monitor or TV) \\
        \hline
        Acuity(min) & acuity of users in arc minutes (1: standard, 0.6: good, 0.13: very good) \\
        \hline
        Distance(mm) & viewing distance for this display \\
        \hline
        PixelMin(mm) & the smallest distance for two pixels to be distinguished by users (2 x distance x tan(acuity/2)) \\
        \hline
        ObsMpix & the maximum observable resolution in mega-pixels (Mpix) = ObsX x ObsY \\
        \hline
        ObsX & the maximum observable resolution in X (number of pixels than users can distinguish) \\
        \hline
        ObsY & the maximum observable resolution in Y (number of pixels than users can distinguish) \\
        \hline
        BetterThanEye & is this display, in this viewing condition, of higher resolution than the eye can see? (is ObsMpix higher than DispMpix?)\\
        \hline
        DispMpix & the actual display resolution in mega-pixels = Actual X * Actual Y \\
        \hline
        Diagonal(mm) & the physical display size in mm diagonally \\
        \hline
        Aspect & the display aspect ratio (often 4:3 or 16:9 FHD)  \\
        \hline
        ActualX & the physical number of pixels in X on the display \\
        \hline
        ActualY & the physical number of pixels in Y on the display \\
        \hline
        Diagonal (pix) & the physical display size in pixel diagonally \\
        \hline
    \end{tabular}
    % \caption{Parameters contained within the dataset}
    % \label{tab:my_label}
\end{table}

\subsection{Gestalt principles and safe colour scale}

Visualize how each display’s resolution compare to the eye’s observable resolution at the given visual acuities. Highlight visually where display resolution exceeds the eye’s observable maximum resolution.\\

Find a way to highlight the main message(s) in the data.\\

\section{Part 2}

Part two is about implementing graphic design principles and colour safe scale on your previous work.

\subsection{Gestalt principles}

Take any one report page from practicals 2 and 3. Duplicate it and apply at least two of the Gestalt principles or graphic design principles discussed during lectures.\\

Seek feedback from a demonstrator once you have completed your work.

\subsection{Safe colour scale}

Take any one report page from practicals 2 and 3. Duplicate it and apply colour safe colours.\\

Tips:
\begin{itemize}
    \item To update all the colours of a dashboard select the 'View' menu and create a custom theme.
\end{itemize}

Once again seek feedback from a demonstrator once you have completed your work.\\
\\
\\
You have now completed all the practical exercises. The next practical sessions will be dedicated to the coursework.

\end{document}